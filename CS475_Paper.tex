%IEEE
\documentclass[conference, 12pt]{IEEEtran}
\usepackage{cite}
\usepackage{amsmath,amssymb,amsfonts}
\usepackage{algorithmic}
\usepackage{graphicx}
\usepackage{textcomp}
\usepackage{xcolor}
\usepackage{hyperref}
\def\BibTeX{{\rm B\kern-.05em{\sc i\kern-.025em b}\kern-.08em
    T\kern-.1667em\lower.7ex\hbox{E}\kern-.125emX}}
\begin{document}

\title{Sockets}

\author{\IEEEauthorblockN{Davis Mariotti}
\IEEEauthorblockA{\textit{Computer Science Department} \\
\textit{Whitworth University}\\
Spokane, USA \\
dmariotti19@my.whitworth.edu}
\and
\IEEEauthorblockN{Michael Gamlem III}
\IEEEauthorblockA{\textit{Computer Science Department} \\
\textit{Whitworth University}\\
Spokane, USA \\
mgamlem19@my.whitworth.edu}
\and
\IEEEauthorblockN{Utsal Shrestha}
\IEEEauthorblockA{\textit{Computer Science Department} \\
\textit{Whitworth University}\\
Spokane, USA \\
ushrestha20@my.whitworth.edu}
}

\maketitle

\begin{abstract}
    Lorem ipsum dolor sit amet, consectetur adipiscing elit. Curabitur pellentesque mauris tellus, a facilisis metus congue et. Nullam gravida laoreet justo, auctor rutrum nibh porttitor nec. Suspendisse potenti. Nam in tincidunt nibh. Pellentesque habitant morbi tristique senectus et netus et malesuada fames ac turpis egestas. Morbi enim ex, dapibus efficitur lorem sit amet, maximus dictum eros. 
\end{abstract}

\begin{IEEEkeywords}
Socket, IP Address, Port, Pipelines
\end{IEEEkeywords}

\section{Introduction}
Modern computer networks rely on many technologies to function efficiently and effectively. The complex system is made of many smaller components each functioning to facilitate communication between two computer endpoints. Among the smaller components is the socket.

Non-common uses

It is attackable?

Can be used in creative ways

Is it the best?


\section{Definition}
A socket is not a physical piece of the computer, but rather, it is the idea that subdivisions can be made on the network interface to separate traffic and route it correctly to where it needs to go. A socket consists of an Internet Protocol (IP) address and a port number. It is commonly written in the form of 192.168.0.1:8080 (Goralski 52). Some port numbers, such as 80 and 443, are reserved for specific applications or communication types while others are free for application programmers to utilize. Both IP Addresses and port numbers are “managed” by many local organizations, however the \href{https://www.iana.org/}{Internet Assigned Numbers Authority} (IANA) is commonly regarded as one of the most accurate authority for reference purposes.

\section{Uses}
Lorem ipsum dolor sit amet, consectetur adipiscing elit. Curabitur pellentesque mauris tellus, a facilisis metus congue et. Nullam gravida laoreet justo, auctor rutrum nibh porttitor nec. Suspendisse potenti. Nam in tincidunt nibh. Pellentesque habitant morbi tristique senectus et netus et malesuada fames ac turpis egestas.

\subsection{General Programming}
Lorem ipsum dolor sit amet, consectetur adipiscing elit. Curabitur pellentesque mauris tellus, a facilisis metus congue et. Nullam gravida laoreet justo, auctor rutrum nibh porttitor nec. Suspendisse potenti. Nam in tincidunt nibh. Pellentesque habitant morbi tristique senectus et netus et malesuada fames ac turpis egestas. Morbi enim ex, dapibus efficitur lorem sit amet, maximus dictum eros.

\subsection{Interprocess Communication}
Lorem ipsum dolor sit amet, consectetur adipiscing elit. Curabitur pellentesque mauris tellus, a facilisis metus congue et. Nullam gravida laoreet justo, auctor rutrum nibh porttitor nec. Suspendisse potenti. Nam in tincidunt nibh. Pellentesque habitant morbi tristique senectus et netus et malesuada fames ac turpis egestas. Morbi enim ex, dapibus efficitur lorem sit amet, maximus dictum eros.

\subsection{Internet Applications}
Lorem ipsum dolor sit amet, consectetur adipiscing elit. Curabitur pellentesque mauris tellus, a facilisis metus congue et. Nullam gravida laoreet justo, auctor rutrum nibh porttitor nec. Suspendisse potenti. Nam in tincidunt nibh. Pellentesque habitant morbi tristique senectus et netus et malesuada fames ac turpis egestas. Morbi enim ex, dapibus efficitur lorem sit amet, maximus dictum eros. 

\subsection{Distributed Computing}
Lorem ipsum dolor sit amet, consectetur adipiscing elit. Curabitur pellentesque mauris tellus, a facilisis metus congue et. Nullam gravida laoreet justo, auctor rutrum nibh porttitor nec. Suspendisse potenti. Nam in tincidunt nibh. Pellentesque habitant morbi tristique senectus et netus et malesuada fames ac turpis egestas. Morbi enim ex, dapibus efficitur lorem sit amet, maximus dictum eros. 

\section{Master/Slave Relationships}
Lorem ipsum dolor sit amet, consectetur adipiscing elit. Curabitur pellentesque mauris tellus, a facilisis metus congue et. Nullam gravida laoreet justo, auctor rutrum nibh porttitor nec. Suspendisse potenti. Nam in tincidunt nibh. Pellentesque habitant morbi tristique senectus et netus et malesuada fames ac turpis egestas. Morbi enim ex, dapibus efficitur lorem sit amet, maximus dictum eros. 

\section{Security}
As with most aspects of the internet today, security is an issue that cannot be ignored when discussing web sockets. Even though the standard has been around for decades, it has not been reliably or adequately updated to protect against modern security threats. This is due to many complicating factors that are outside the scope of this paper. However, below are some examples of both commonly exploited and uncommonly exploited vulnerabilities in the socket communication protocol as well as potential experimental solutions.

\subsection{Vulnerabilities}
There is no security built into the socket protocol. This means that all applications that use sockets must be designed to protect against any and all possible attack vectors. Some protections are given by host operating systems, however they are not all-inclusive and not always sufficient.

\subsection{Exploits}
Lorem ipsum dolor sit amet, consectetur adipiscing elit. Curabitur pellentesque mauris tellus, a facilisis metus congue et. Nullam gravida laoreet justo, auctor rutrum nibh porttitor nec. Suspendisse potenti. Nam in tincidunt nibh. Pellentesque habitant morbi tristique senectus et netus et malesuada fames ac turpis egestas. Morbi enim ex, dapibus efficitur lorem sit amet, maximus dictum eros. 

\subsection{Potential Solutions}
Lorem ipsum dolor sit amet, consectetur adipiscing elit. Curabitur pellentesque mauris tellus, a facilisis metus congue et. Nullam gravida laoreet justo, auctor rutrum nibh porttitor nec. Suspendisse potenti. Nam in tincidunt nibh. Pellentesque habitant morbi tristique senectus et netus et malesuada fames ac turpis egestas. Morbi enim ex, dapibus efficitur lorem sit amet, maximus dictum eros. 

\section{Experimental Uses}
Lorem ipsum dolor sit amet, consectetur adipiscing elit. Curabitur pellentesque mauris tellus, a facilisis metus congue et. Nullam gravida laoreet justo, auctor rutrum nibh porttitor nec. Suspendisse potenti. Nam in tincidunt nibh. Pellentesque habitant morbi tristique senectus et netus et malesuada fames ac turpis egestas. Morbi enim ex, dapibus efficitur lorem sit amet, maximus dictum eros. 

\section{Comparison}
Lorem ipsum dolor sit amet, consectetur adipiscing elit. Curabitur pellentesque mauris tellus, a facilisis metus congue et. Nullam gravida laoreet justo, auctor rutrum nibh porttitor nec. Suspendisse potenti. Nam in tincidunt nibh. Pellentesque habitant morbi tristique senectus et netus et malesuada fames ac turpis egestas. Morbi enim ex, dapibus efficitur lorem sit amet, maximus dictum eros. 

\subsection{Pipes}
Lorem ipsum dolor sit amet, consectetur adipiscing elit. Curabitur pellentesque mauris tellus, a facilisis metus congue et. Nullam gravida laoreet justo, auctor rutrum nibh porttitor nec. Suspendisse potenti. Nam in tincidunt nibh. Pellentesque habitant morbi tristique senectus et netus et malesuada fames ac turpis egestas. Morbi enim ex, dapibus efficitur lorem sit amet, maximus dictum eros. 

\subsection{Remote Procedure Calls (RPC)}
Lorem ipsum dolor sit amet, consectetur adipiscing elit. Curabitur pellentesque mauris tellus, a facilisis metus congue et. Nullam gravida laoreet justo, auctor rutrum nibh porttitor nec. Suspendisse potenti. Nam in tincidunt nibh. Pellentesque habitant morbi tristique senectus et netus et malesuada fames ac turpis egestas. Morbi enim ex, dapibus efficitur lorem sit amet, maximus dictum eros. 

\section{Conclusion}
Lorem ipsum dolor sit amet, consectetur adipiscing elit. Curabitur pellentesque mauris tellus, a facilisis metus congue et. Nullam gravida laoreet justo, auctor rutrum nibh porttitor nec. Suspendisse potenti. Nam in tincidunt nibh. Pellentesque habitant morbi tristique senectus et netus et malesuada fames ac turpis egestas. Morbi enim ex, dapibus efficitur lorem sit amet, maximus dictum eros. Lorem ipsum dolor sit amet, consectetur adipiscing elit. Curabitur pellentesque mauris tellus, a facilisis metus congue et. Nullam gravida laoreet justo, auctor rutrum nibh porttitor nec. Suspendisse potenti. Nam in tincidunt nibh. Pellentesque habitant morbi tristique senectus et netus et malesuada fames ac turpis egestas. Morbi enim ex, dapibus efficitur lorem sit amet, maximus dictum eros. 

\section*{References}

Please number citations consecutively within brackets \cite{b1}. The 
sentence punctuation follows the bracket \cite{b2}. Refer simply to the reference 
number, as in \cite{b3}---do not use ``Ref. \cite{b3}'' or ``reference \cite{b3}'' except at 
the beginning of a sentence: ``Reference \cite{b3} was the first $\ldots$''

Number footnotes separately in superscripts. Place the actual footnote at 
the bottom of the column in which it was cited. Do not put footnotes in the 
abstract or reference list. Use letters for table footnotes.

Unless there are six authors or more give all authors' names; do not use 
``et al.''. Papers that have not been published, even if they have been 
submitted for publication, should be cited as ``unpublished'' \cite{b4}. Papers 
that have been accepted for publication should be cited as ``in press'' \cite{b5}. 
Capitalize only the first word in a paper title, except for proper nouns and 
element symbols.

For papers published in translation journals, please give the English 
citation first, followed by the original foreign-language citation \cite{b6}.

\begin{thebibliography}{00}
\bibitem{b1} G. Eason, B. Noble, and I. N. Sneddon, ``On certain integrals of Lipschitz-Hankel type involving products of Bessel functions,'' Phil. Trans. Roy. Soc. London, vol. A247, pp. 529--551, April 1955.
\bibitem{b2} J. Clerk Maxwell, A Treatise on Electricity and Magnetism, 3rd ed., vol. 2. Oxford: Clarendon, 1892, pp.68--73.
\bibitem{b3} I. S. Jacobs and C. P. Bean, ``Fine particles, thin films and exchange anisotropy,'' in Magnetism, vol. III, G. T. Rado and H. Suhl, Eds. New York: Academic, 1963, pp. 271--350.
\bibitem{b4} K. Elissa, ``Title of paper if known,'' unpublished.
\bibitem{b5} R. Nicole, ``Title of paper with only first word capitalized,'' J. Name Stand. Abbrev., in press.
\bibitem{b6} Y. Yorozu, M. Hirano, K. Oka, and Y. Tagawa, ``Electron spectroscopy studies on magneto-optical media and plastic substrate interface,'' IEEE Transl. J. Magn. Japan, vol. 2, pp. 740--741, August 1987 [Digests 9th Annual Conf. Magnetics Japan, p. 301, 1982].
\bibitem{b7} M. Young, The Technical Writer's Handbook. Mill Valley, CA: University Science, 1989.
\end{thebibliography}

\end{document}
